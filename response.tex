\documentclass{article}


\usepackage[T1]{fontenc}
\usepackage[caption=false,font=footnotesize]{subfig}
\usepackage{adjustbox}
\usepackage{amsfonts}
\usepackage{amsmath}
\usepackage{amssymb}
\usepackage{amsthm}
\usepackage{cite}
\usepackage{fullpage}
\usepackage{hyperref}
\usepackage{mathtools}
\usepackage{microtype}
\usepackage{multirow}
\usepackage{pdfpages}
\usepackage{pgfplots}
\usepackage{siunitx}


\DeclareSIUnit{\belm}{Bm}
\DeclareSIUnit{\dBm}{\deci\belm}
\DeclareSIUnit{\beli}{Bi}
\DeclareSIUnit{\dBi}{\deci\beli}

\newcounter{reviewer}
\setcounter{reviewer}{0}
\newcounter{point}[reviewer]
\setcounter{point}{0}
\newcounter{response}[reviewer]
\setcounter{response}{0}

\let\svbibcite\bibcite
\def\bibcite#1#2{\svbibcite{#1}{R#2}}
\makeatletter
\let\svbiblabel\@biblabel
\def\@biblabel#1{\svbiblabel{R#1}}
\makeatother

\newcommand{\reviewersection}
	{\stepcounter{reviewer} \bigskip \hrule \section*{Reviewer \thereviewer}}

\renewcommand{\thepoint}
	{\thereviewer.\arabic{point}}

\renewcommand{\theresponse}
	{\thereviewer.\arabic{response}}

\newenvironment{point}
	{\refstepcounter{point} \bigskip \noindent {\textbf{Comment~\thepoint} } ---\ \itshape}
	{\par}

\newenvironment{response}
	{\refstepcounter{response} \medskip \noindent \textbf{Response:}\ }
	{\medskip}


\begin{document}
	\includepdf{letter.pdf}

	\begin{reviewersection}
		\begin{point}
			This paper assumed that the perfect channel state information (CSI) of the whole system is available at the base station. However, since the IRS is in general not equipped with a radio frequency chain, the accurate CSI of the reflecting links established by the IRS is very challenging to obtain. The reviewer wonders if the proposed algorithm can also be applied to a case where the imperfect CSI of the network is available at the base station.
			\label{pt:1.1}
		\end{point}

		\begin{response}
			Thank you for pointing out the CSI acquisition issue regarding the cascaded AP-IRS-UE link. The proposed passive beamforming algorithms rely on such cascaded CSIT, and recent researches proposed element-wise on/off switching [25], joint training sequence and reflection pattern design \cite{Kang2020} and compressed sensing \cite{Wang2020} techniques to solve this issue. Those references have been cited in the literature review. To address the reviewer' comments, we consider an imperfect CSIT model where the estimation of the cascaded link at subband $n$ is
			\begin{equation}
				\hat{\boldsymbol{V}}_{n} = \boldsymbol{V}_{n} + \tilde{\boldsymbol{V}}_{n}
			\end{equation}
			where $\tilde{\boldsymbol{V}}_{n}$ is the estimation error with entries following i.i.d. CSCG distribution $\mathcal{CN}(0, \epsilon_{n}^2)$. Simulation result in Fig.~\ref{fi:re_csi} demonstrates the robustness of the proposed passive beamforming algorithm to cascaded CSIT inaccuracy for broadband SWIPT with different number of IRS elements.

			\begin{figure}[!h]
				\centering
				\resizebox{0.45\columnwidth}{!}{
					% This file was created by matlab2tikz.
%
%The latest updates can be retrieved from
%  http://www.mathworks.com/matlabcentral/fileexchange/22022-matlab2tikz-matlab2tikz
%where you can also make suggestions and rate matlab2tikz.
%
\definecolor{mycolor1}{rgb}{0.00000,0.44706,0.74118}%
\definecolor{mycolor2}{rgb}{0.85098,0.32549,0.09804}%
%
\begin{tikzpicture}

\begin{axis}[%
width=4.036in,
height=3.396in,
at={(0.677in,0.458in)},
scale only axis,
xmin=0,
xlabel style={font=\color{white!15!black}},
xlabel={Per-subband rate [bps/Hz]},
ymin=0,
ylabel style={font=\color{white!15!black}},
ylabel={Output DC current [$\mu$A]},
axis background/.style={fill=white},
xmajorgrids,
ymajorgrids,
legend style={legend cell align=left, align=left, draw=white!15!black},
title style={font=\huge}, label style={font=\huge}, ticklabel style={font=\LARGE}, legend style={font=\LARGE}
]
\addplot [color=mycolor1, line width=2.0pt, mark=o, mark options={solid, mycolor1}]
  table[row sep=crcr]{%
6.64939284863596	3.86799927313902e-15\\
6.29940939889854	6.46379995175324\\
5.94943506824246	15.2265422150975\\
5.59945327378938	24.840970562481\\
5.24947353564041	34.6014705909588\\
4.89949155984587	44.0661923823857\\
4.54950691184103	54.8413499189593\\
4.19951832538194	68.0551239585303\\
3.84953342975971	82.6311240388772\\
3.49955922805664	98.1305436453771\\
3.14960264995	114.569198945373\\
2.79963608300121	131.610218098534\\
2.44966493789768	149.122082623992\\
2.099703172694	167.2364928282\\
1.74973096407342	185.737977233467\\
1.3997747538097	204.526689299556\\
1.04982212170582	223.468667605056\\
0.699878776764884	242.19178355121\\
0.349922243009016	261.285832920743\\
1.69114633332955e-08	292.44773255779\\
};
\addlegendentry{$\epsilon_n^2 = 0.1 \Lambda_I\Lambda_R$ $(L = 20)$}

\addplot [color=mycolor1, dashed, line width=2.0pt, mark=+, mark options={solid, mycolor1}]
  table[row sep=crcr]{%
6.64839269305236	3.86559075420031e-15\\
6.29837473291061	6.46021025412597\\
5.94836460787289	15.2158840697973\\
5.59837132806164	24.8201257397726\\
5.24839132535847	34.56402928778\\
4.89844397917428	44.01544473559\\
4.54847878623829	54.7536556651023\\
4.19845943220864	67.9548074717084\\
3.84849362284492	82.5203893053185\\
3.4985644932664	98.1392912698387\\
3.14858122529503	114.514934550592\\
2.79866120325396	131.586332758405\\
2.44867382531395	149.105867448946\\
2.09862754265233	167.131285401495\\
1.74861265107306	185.481788732111\\
1.39867030714913	204.315995574498\\
1.04871034542171	223.262943600108\\
0.698810524825426	241.853394344499\\
0.349213725556016	260.869851541717\\
1.41918958977462e-08	291.982171236655\\
};
\addlegendentry{$\epsilon_n^2 = 1 \Lambda_I\Lambda_R$ $(L = 20)$}

\addplot [color=mycolor1, dotted, line width=2.0pt, mark=square, mark options={solid, mycolor1}]
  table[row sep=crcr]{%
6.63775657957584	3.84238653864391e-15\\
6.28733056873462	6.42657402846615\\
5.93700398997769	15.1251413991258\\
5.58683920539835	24.6560014809727\\
5.23681133949828	34.3128887557631\\
4.88693431821189	43.6337995643153\\
4.5370020862379	54.3283077232771\\
4.18713241315668	67.4633153219984\\
3.83734801658291	81.7985083705019\\
3.48754706515635	97.1579700545024\\
3.13754754237405	113.657877917017\\
2.78785469849976	130.568049342272\\
2.43836746816564	147.90762343092\\
2.08879883582737	165.776860334261\\
1.73926114237747	183.732157373243\\
1.39026682340013	202.278302184522\\
1.04155358950709	221.152761323149\\
0.692726536196467	239.399287692999\\
0.344834647570243	257.920650162905\\
1.70545288615e-07	288.594855048297\\
};
\addlegendentry{$\epsilon_n^2 = 10 \Lambda_I\Lambda_R$ $(L = 20)$}

\addplot [color=mycolor1, dashdotted, line width=2.0pt, mark=x, mark options={solid, mycolor1}]
  table[row sep=crcr]{%
5.91038107497456	0\\
5.59930930054115	3.86353936114123\\
5.28823759202524	8.91605512593181\\
4.97716663533409	14.5255901096671\\
4.66609566451225	20.3443597970165\\
4.35502316425296	26.099932203759\\
4.04395052156944	32.1644862531963\\
3.73287755510416	39.1985215584596\\
3.42180809503495	47.0596062444363\\
3.11074268142394	55.5366717369099\\
2.79967797278857	64.271729749289\\
2.48861428897231	73.1021332445637\\
2.17755054173136	82.0431816512312\\
1.86649607216554	91.1794230286471\\
1.55543975003996	100.319802829633\\
1.24437991926154	109.453913896588\\
0.933310553047552	118.712870265117\\
0.622245496068856	128.170077631544\\
0.311163760912639	138.472980445164\\
4.49481416766612e-05	155.719014500102\\
};
\addlegendentry{Random IRS $(L = 20)$}

\addplot [color=mycolor2, line width=2.0pt, mark=triangle, mark options={solid, mycolor2}]
  table[row sep=crcr]{%
7.07787909023998	4.94138960307007e-15\\
6.70537113411751	9.23355964959484\\
6.33286476330775	22.0224771601093\\
5.96036094997897	35.8887008214907\\
5.58785958648164	49.5483033029588\\
5.21536732763876	63.6614178083605\\
4.84286273291985	80.1232593524828\\
4.47035180495791	100.685017083096\\
4.09785624803477	124.09293586292\\
3.72537040734482	149.000295449576\\
3.35287124332091	174.95471079264\\
2.98034333740517	202.22289781976\\
2.6078223231037	230.085218651969\\
2.23524742037047	258.747607598535\\
1.86268742226957	286.520869205691\\
1.49010277736294	316.419855822664\\
1.11750850242133	347.929674714296\\
0.74497327353138	378.374599339219\\
0.372480890032973	408.89454030671\\
1.55466276994436e-08	456.17282736323\\
};
\addlegendentry{$\epsilon_n^2 = 0.1 \Lambda_I\Lambda_R$ $(L = 40)$}

\addplot [color=mycolor2, dashed, line width=2.0pt, mark=triangle, mark options={solid, rotate=180, mycolor2}]
  table[row sep=crcr]{%
7.07610760856136	4.93689051028324e-15\\
6.7035552538886	9.22383907392057\\
6.33100623818965	21.9978773107847\\
5.95847179292927	35.8511811699964\\
5.58594736969153	49.494563510267\\
5.21343146109728	63.6040854940904\\
4.84086189380499	80.0543251077591\\
4.46828850446026	100.610272989299\\
4.09572174514724	124.044630963121\\
3.7232028146498	148.959868961671\\
3.35073767011254	174.944071280813\\
2.97828276170064	202.189976673296\\
2.60570103667946	229.963851144625\\
2.23314252940547	258.300958640263\\
1.86070713531808	286.116436674294\\
1.48809015824535	316.019526822644\\
1.1154830357114	347.372670013874\\
0.743252462715435	377.538070500386\\
0.371327861209583	407.799645838797\\
1.37839875002086e-08	454.907393850181\\
};
\addlegendentry{$\epsilon_n^2 = 1 \Lambda_I\Lambda_R$ $(L = 40)$}

\addplot [color=mycolor2, dotted, line width=2.0pt, mark=triangle, mark options={solid, rotate=270, mycolor2}]
  table[row sep=crcr]{%
7.05878767163745	4.89239807062337e-15\\
6.68534916875118	9.14451101107585\\
6.31198754798203	21.7811699737901\\
5.93885744210493	35.4559583157413\\
5.56574305927843	48.8751197940232\\
5.19281113165521	62.6636213978829\\
4.8196492190202	79.0489538046423\\
4.44642434168568	99.395826421139\\
4.07347269967762	122.445950961453\\
3.70055642803626	147.011094869209\\
3.32815521198334	172.491159805627\\
2.95618310045654	199.002011532474\\
2.58442609683226	226.566091887233\\
2.21271579189769	254.240654550101\\
1.84122672422511	282.104806878312\\
1.47036604461841	311.15529780851\\
1.09999813005235	340.963269610684\\
0.730230712695192	369.774437200029\\
0.362188501691438	398.712780818864\\
2.0962902357931e-08	444.58185250445\\
};
\addlegendentry{$\epsilon_n^2 = 10 \Lambda_I\Lambda_R$ $(L = 40)$}

\addplot [color=mycolor2, dashdotted, line width=2.0pt, mark=triangle, mark options={solid, rotate=90, mycolor2}]
  table[row sep=crcr]{%
5.8172393176653	0\\
5.51107126293838	3.65768945385545\\
5.20490020183974	8.36018473431415\\
4.8987323939638	13.5198789943025\\
4.59256222233627	18.8298662869982\\
4.28639174419897	23.9724993035307\\
3.98022181222138	29.9905453448942\\
3.67405472858631	36.4127376987519\\
3.36788893363125	43.6107464338869\\
3.06172397158839	51.4098711831526\\
2.75555853462649	59.6591803672519\\
2.44939769370914	68.2197621962284\\
2.1432357521821	76.8992916099036\\
1.83707712465568	85.5819880011987\\
1.53092120905368	94.2809528078039\\
1.22476765836412	103.062487037657\\
0.91860894006099	111.962997153661\\
0.612438091364928	121.099923749709\\
0.306259382514816	130.884836552661\\
0.000189783748217323	147.436482186797\\
};
\addlegendentry{Random IRS $(L = 40)$}

\end{axis}
\end{tikzpicture}%
				}
				\label{fi:re_csi}
				\caption{Average R-E region with imperfect cascaded CSIT for $M=1$, $N=16$, $L=20$, $\sigma_n^2=\SI{-40}{\dBm}$, $B=\SI{10}{\MHz}$ and $d_{\mathrm{H}}=d_{\mathrm{V}}=\SI{2}{\meter}$.}
			\end{figure}

			\label{re:1.1}
		\end{response}

		\begin{point}
			This paper considered a relatively simple system model where there is only one user in the system. However, in practice, there can be many users co-existing in the SWIPT systems and there may also be two different and independent quality of service requirements: energy harvesting requirement and information decoding requirement. It could be better if the authors can clarify if the proposed algorithm can be employed in a more general case where the base station and the IRS cooperate to serve multiple users in the same time slot.
			\label{pt:1.2}
		\end{point}

		\begin{response}
			Multi-user IRS-aided SWIPT is very interesting and should be investigated in the future, but is left for future work for the reasons explained below.

			First, it is important to remind the reviewer that the problem of joint waveform and beamforming design for a multi-user SWIPT system without IRS has never been studied in the literature. In other words, it would not make sense at this stage to address this multiuser IRS-aided SWIPT since the underlying building block of multiuser SWIPT is not known. Only once the problem of multiuser joint waveform and beamforming for multi-user SWIPT has been addressed, can we investigate the multiuser joint waveform and beamforming design for IRS-aided SWIPT. Those problems could be addressed in future research, but are not the scope of this paper.

			Second, the emphasis is on single-user because we believe a proper understanding of the single-user case is crucial before jumping into multi-user scenarios. Our modeling of SWIPT is not "simple": not only we transmit both information and power, we deal with joint space and frequency optimization through a joint beamforming and waveform problem that is challenging due to the nonlinearity of the rectifier that induces coupling among frequencies components. This completely contrasts with any existing IRS-aided SWIPT work as discussed in the introduction [31 -- 33].

			Third, multi-user WPT leads to its own set of challenges compared single-user WPT. Taking an non-IRS-aided WPT, frequency and spatial domains can be decoupled in single-user case, while they cannot be decoupled in the multi-user case, as shown in [8, 36]. Consequently, single-user and multi-user WPT are designed differently in [8, 36]. Though the SWIPT case has not been studied, the same set of problems is expected to occur between single user and multiuser case and would be even more challenging since communication is delivered simultaneously with power. Adding IRS into the picture would make it even more complicated.
			\label{re:1.2}
		\end{response}

		\begin{point}
			It is well known that after employing semidefinite programming for handling the phase shift matrix at the IRS, it is very unlikely to obtain a rank-one phase shift matrix without any further modification. The reviewer notices that the authors proposed the Gaussian randomization method to ensure a rank-one solution. Also, the convergence of the proposed overall algorithm also relies on the unit-rank solution. Therefore, to make this paper more comprehensive and convincing, it is suggested to provide more results (such as figures, tables, and data analysis) in the simulation part to show that the rank-one solution can always be obtained even without applying the Gaussian randomization. As this is a very important and interesting conclusion to the colleagues working in the same area, it could be better if the authors can further discuss, interpret, and clarify this in a remark.
			\label{pt:1.3}
		\end{point}

		\begin{response}
			Thank you for raising this issue. The rank-\num{1} property of the IRS matrix $\boldsymbol{\Phi}$ to the relaxed problem (25a) -- (25d) is suggested by the CVX software [39] under different configurations. Although the strict convergence and local optimality of the BCD Algorithm~3 relies on this conclusion, it is intricate to provide a mathematical proof due to the coupled objective (24) and the log sum constraint (25b). Instead, we define the eigen ratio as $\nu=\max_l\lambda_l(\boldsymbol{\Phi})/\sum_l\lambda_l(\boldsymbol{\Phi})$ where $\lambda_l(\boldsymbol{\Phi})$ is the $l$-th eigenvalue of $\boldsymbol{\Phi}$. Since $\nu=1$ means $\boldsymbol{\Phi}$ is rank-\num{1}, we claim $\boldsymbol{\Phi}$ is approximately rank-\num{1} when $1-\nu \approx 0$. Table~\ref{ta:eigen_ratio} shows the maximum value of $1-\nu$ over \num{20} R-E samples of \num{300} tested channel realizations for both BCD algorithms. With reasonable precision, we conclude that $\boldsymbol{\Phi}$ is always rank-\num{1} under different configurations for all tested channel realizations. It suggests that the rank constraint (25e) can be relaxed with negligible performance loss, $\bar{\boldsymbol{\phi}}$ can be directly obtained through eigen decomposition, and the assumption of Proposition~3 always hold such that the BCD Algorithm~3 converges to local optimal points.

			\begin{table*}[!h]
				\caption{The maximum value of $1-\nu$ over all R-E samples during BCD and LC-BCD algorithms for all tested channel realizations}
				\label{ta:eigen_ratio}
				\centering
				\resizebox{\columnwidth}{!}{
					\begin{tabular}{|c|c|c|c|c|c|c|c|c|c|c|c|c|}
						\hline
						\multicolumn{2}{|c|}{\multirow{2}{*}{}} & \multicolumn{3}{c|}{$M$} & \multicolumn{3}{c|}{$N$} & \multicolumn{3}{c|}{$L$} & \multicolumn{2}{c|}{$B$} \\ \cline{3-13}
						\multicolumn{2}{|c|}{} & \num{1} & \num{2} & \num{4} & \num{1} & \num{4} & \num{16} & \num{20} & \num{40} & \num{80} & \SI{1}{\MHz} & \SI{10}{\MHz} \\ \hline
						$\max(1 - \nu)$ & BCD & \num{5.5511} & \num{6.6613} & \num{6.6613} & \num{5.5511} & \num{6.6613} & \num{5.5511} & \num{6.6613} & \num{5.5511} & \num{6.6613} & \num{6.6613} & \num{5.5511} \\ \cline{2-13}
						$[\times \num{e-16}]$ & LC & \num{6.6613} & \num{6.6613} & \num{6.6613} & \num{5.5511} & \num{6.6613} & \num{5.5511} & \num{6.6613} & \num{8.8818} & \num{6.6613} & \num{6.6613} & \num{5.5511} \\ \hline
					\end{tabular}
				}
			\end{table*}

			\label{re:1.3}
		\end{response}

		\begin{point}
			The figures in the current version are relatively small, it is suggested to provide larger figures to help the readers better understand the results.
			\label{pt:1.4}
		\end{point}

		\begin{response}
			We sincerely apologize for the inconvenience. The figures would be amplified in future versions.
			\label{re:1.4}
		\end{response}

		\begin{point}
			In the current version, the authors claimed that by applying semidefinite relaxation and omitting the rank-one constraint, the performance loss is negligible. The reviewer wonders if this is because of the relatively simple system model, as there is only one single user who has both power and information requirements. It could be better if the authors can further discuss this issue for a more general multiple user scenario.
			\label{pt:1.5}
		\end{point}

		\begin{response}
			We agree with the reviewer that the relaxation on the rank-one constraint deserves more attention. However, as discussed in Response~\ref{re:1.2}, multi-user IRS-aided SWIPT is out the scope of this paper and we would like to provide some general ideas to address this issue. For any passive beamforming problem with unit-rank constraint (which is non-convex), we can either relax the constraint and use Gaussian randomization method to obtain a high-quality solution [38], or replace the unit-rank constraint $\mathrm{rank}(\boldsymbol{\Phi})=1$ with the constraint on largest singular value $\mathrm{Tr}(\boldsymbol{\Phi})-\sigma_{\max} (\boldsymbol{\Phi}) \le 0$ then solve the problem by the penalty method [33], \cite{Wu2021}. Both techniques are expected to provide a unit-rank solution with performance very close to the original solution.
			\label{re:1.5}
		\end{response}
	\end{reviewersection}

	\begin{reviewersection}
		\begin{point}
			New RIS models are now being adopted as in [24] where it has been shown that the reflected signals depend on the direction of the arriving signal and this needs to be included in the analysis for realistic quantification.
			\label{pt:2.1}
		\end{point}

		\begin{response}
			Thank you for sharing this paper. It investigated the impact of non-zero effective resistance on the reflection pattern and pointed out that the amplitude of the reflection coefficient depends on the phase shift forced on the incoming signal when power dissipation is considered at the IRS. It also proposed an analytical IRS model together with an BCD algorithm to maximize the achievable rate by passive beamforming. Simulation results emphasized the importance of modeling such a relationship in practical IRS design. There exist various refined models of IRS in the literature, however, we have finally decided to use the most common and simplest IRS model at the current stage to reduce the design complexity and provide a primary benchmark for practical IRS-aided SWIPT.
			\label{re:2.1}
		\end{response}

		\begin{point}
			Why is MRT considered as precoder by (27) rather than optimizing it? Is it globally optimal too?
			\label{pt:2.2}
		\end{point}

		\begin{response}
			In the single-user scenario, the global optimal information and power precoders coincide at MRT. To prove this, we decouple the waveform in the spatial and frequency domains by
			\begin{equation}\label{eq:w}
				\boldsymbol{w}_{\mathrm{I/P}, n} = s_{\mathrm{I/P}, n} \boldsymbol{b}_{\mathrm{I/P}, n}
			\end{equation}
			where $s_{\mathrm{I/P},n}$ denotes the amplitude of modulated/multisine waveform at tone $n$, and $\boldsymbol{b}_{\mathrm{I/P}, n}$ denotes the information/power precoder. The MRT precoder at subband $n$ is given by
			\begin{equation}\label{eq:b}
				\boldsymbol{b}_{\mathrm{I/P}, n}^\star = \frac{\boldsymbol{h}_n}{\lVert{\boldsymbol{h}_n}\rVert}
			\end{equation}

			From the perspective of WIT, the MRT precoder maximizes $\lvert{\boldsymbol{h}_{n}^H \boldsymbol{w}_{\mathrm{I}, n}}\rvert = \lVert{\boldsymbol{h}_{n}}\rVert s_{\mathrm{I}, n}$ thus maximizes the rate (8). From the perspective of WPT, the MRT precoder maximizes $(\boldsymbol{h}_{n}^H \boldsymbol{w}_{\mathrm{I/P}, n})(\boldsymbol{h}_{n}^H \boldsymbol{w}_{\mathrm{I/P}, n})^* = \lVert{\boldsymbol{h}_{n}}\rVert^2 s_{\mathrm{I/P}, n}^2$ thus maximizes the second and fourth order DC terms (11) -- (14). Hence, MRT is the global optimal active precoder and no dedicated energy beams are required in the single-user SWIPT. We have separated the active beamforming design from the waveform design in the revised manuscript to clarify this point.
			\label{re:2.2}
		\end{response}

		\begin{point}
			Some strong assumptions like perfect CSI availability limit the practical utility of the proposed analytical results.
			\label{pt:2.3}
		\end{point}

		\begin{response}
			The reviewer is referred to Response \ref{re:1.1}. Indeed, the assumption of perfect CSIT is very ideal and the existing channel estimation protocols may not provide decent results in practice. We have investigated the impact of CSIT estimation error of the cascaded AP-IRS-UE link on the R-E performance, and Fig.~\ref{fi:re_csi} shows that the proposed algorithms are robust to CSIT inaccuracy.
			\label{re:2.3}
		\end{response}

		\begin{point}
			All the assumptions and relaxations adopted used in the derivation of results as in (24) need to be explicitly mentioned along with appropriate justification for the same.
			\label{pt:2.4}
		\end{point}

		\begin{response}
			We appreciate your suggestion and have revised the manuscript correspondingly. The original objective function (20) is differentiable and non-concave in $\mathbb{C}^{4N - 2}$, and we approximate (linearize) the second-order terms by the first-order Taylor expansions (21) -- (23) to formulate SCA problems of the original passive beamforming problem (i.e. maximize (20) s.t. (25b) -- (25e)). The objective affine function (24) is obtained by plugging (21) -- (23) into (20). It satisfies $\tilde{z}(\boldsymbol{\Phi}^{(i)}, \boldsymbol{\Phi}^{(i)}) = z(\boldsymbol{\Phi}^{(i)}) \ge \tilde{z}(\boldsymbol{\Phi}^{(i)}, \boldsymbol{\Phi}^{(i-1)})$ such that solving (25) iteratively is guaranteed to converge to a local optimal point of the original passive beamforming problem.
			\label{re:2.4}
		\end{response}

		\begin{point}
			Some transformations have been made while solving the original problem, but it has not been explicitly mentioned whether it is equivalent to transformation or not.
			\label{pt:2.5}
		\end{point}

		\begin{response}
			Thank you for the reminder. All transformations are equivalent to their original form and we have mentioned this point explicitly in the revised manuscript.
			\label{re:2.5}
		\end{response}

		\begin{point}
			Are the proposed solutions locally optimal or globally optimal? It is not clear whether the convergence of proposed solution methodologies is local or global? Also, how fast is it?
			\label{pt:2.6}
		\end{point}

		\begin{response}
			Algorithm~1 -- 4 only provide local optimal solutions with analytical/numerical local convergence proof, and the performance indeed depends on the initialization. For the passive beamforming Algorithm~1, we initialize the phase shift of all IRS elements as i.i.d. uniform random variables over $[0, 2\pi)$. For the waveform amplitude and splitting ratio Algorithm~2, we initialize the modulated waveform by the Water-Filling (WF) strategy (37) and the multisine waveform by the Scaled Matched Filter (SMF) scheme (38), and assume $\rho^{(0)}=\bar{\rho}^{(0)}=1$. These parameters are used for general initialization and regulated by the algorithm afterwards. However, as Algorithm~3 only converge to local optimal solutions, few R-E points might be strictly worse (with less rate and energy) than the others especially when $N$ is very large. To address this issue, we draw the R-E boundary from the high-rate low-energy (lower right) points to the low-rate high-energy (upper left) points. If a point is strictly dominated, we discard the candidate, reinitialize Algorithm~3 by the solution at the previous point, then perform the optimization again. For a tolerance of $\epsilon=10^{-8}$, Algorithm~1 -- 4 typically converge within 2, 7, 2, 2 iterations, respectively.
			\label{re:2.6}
		\end{response}

		\begin{point}
			The time complexity of the proposed algorithms, especially involving branch and bound methods, seems to be high especially applications assuming perfectly CSI availability as the coherence times are practically pretty low. So, the authors would like to justify it so that the proposed solution can be obtained over relatively short coherence intervals.
			\label{pt:2.7}
		\end{point}

		\begin{response}
			Problem (25) is not a Semidefinite Programming (SDP) due to the $(t_{\mathrm{I},0}^{(i)} - t_{\mathrm{P},0}^{(i)})^2$ term in (24). Hence, existing complexity analysis tools cannot be direct applied to Algorithm~1 and 4. For Algorithm~2, the computational complexity scales exponentially with the number of subbands [36, 43], but an analytical expression cannot be derived on top of relevant literatures. To facilitate practical SWIPT implementation, we propose two closed-form adaptive waveform amplitude design by combining WF and SMF under TS and PS setups. The optimal waveform design for WIT corresponds to the WF strategy that assigns the amplitude of modulated tone $n$ by
			\begin{equation}\label{eq:wf}
				s_{\mathrm{I}, n} = \sqrt{2\left(\mu - \frac{\sigma_n^2}{P \lVert{\boldsymbol{h}_n}\rVert^2}\right)^+}
			\end{equation}
			where $\mu$ is chosen to satisfy the power constraint $\lVert{\boldsymbol{s}_I}\rVert^2 / 2 \le P$. The closed-form solution can be obtained by iterative power allocation [45] and the details are omitted here. On the other hand, SMF was proposed in [12] as a suboptimal WPT resource allocation scheme that assigns the amplitude of sinewave $n$ by
			\begin{equation}\label{eq:smf}
				s_{\mathrm{P}, n} = \sqrt{\frac{2 P}{\sum_{n=1}^N \lVert{\boldsymbol{h}_n \rVert^{2 \alpha}}}}\lVert{\boldsymbol{h}_n}\rVert^\alpha
			\end{equation}
			where the scaling ratio $\alpha \ge 1$ scales the matched filter to exploit the rectifier nonlinearity. In the low-complexity TS waveform design, modulated waveform \eqref{eq:wf} is used in the data session while multisine waveform \eqref{eq:smf} is used in the energy session. In contrast, the low-complexity PS scheme jointly designs the waveform balancing ratio $\delta$ and splitting ratio $\rho$, and the amplitude of modulated and multisine components are given by
			\begin{align}
				s_{\mathrm{I}, n} &= \sqrt{2(1 - \delta)\left(\mu - \frac{\sigma_n^2}{P \lVert{\boldsymbol{h}_n}\rVert^2}\right)^+} \label{eq:s_i}\\
				s_{\mathrm{P}, n} &= \sqrt{\frac{2 \delta P}{\sum_{n=1}^N \lVert{\boldsymbol{h}_n \rVert^{2 \alpha}}}}\lVert{\boldsymbol{h}_n}\rVert^\alpha \label{eq:s_p}
			\end{align}
			Fig.~\ref{fi:re_noise} -- \ref{fi:re_reflector} demonstrate that the low-complexity BCD algorithm achieves near-optimal performance under different configurations.

			\begin{figure}[!h]
				\centering
				\subfloat[Average noise power\label{fi:re_noise}]{
					\resizebox{0.45\columnwidth}{!}{
						% This file was created by matlab2tikz.
%
%The latest updates can be retrieved from
%  http://www.mathworks.com/matlabcentral/fileexchange/22022-matlab2tikz-matlab2tikz
%where you can also make suggestions and rate matlab2tikz.
%
\definecolor{mycolor1}{rgb}{0.00000,0.44700,0.74100}%
\definecolor{mycolor2}{rgb}{0.85000,0.32500,0.09800}%
\definecolor{mycolor3}{rgb}{0.92900,0.69400,0.12500}%
\definecolor{mycolor4}{rgb}{0.49400,0.18400,0.55600}%
\definecolor{mycolor5}{rgb}{0.46600,0.67400,0.18800}%
%
\begin{tikzpicture}

\begin{axis}[%
width=4.521in,
height=1.575in,
at={(0.758in,2.472in)},
scale only axis,
xmin=0,
xlabel style={font=\color{white!15!black}},
xlabel={Per-subband rate [bps/Hz]},
ymin=0,
ylabel style={font=\color{white!15!black}},
ylabel={Average output DC current [$\mu$A]},
axis background/.style={fill=white},
xmajorgrids,
ymajorgrids,
legend style={legend cell align=left, align=left, draw=white!15!black}
]
\addplot [color=mycolor1, line width=2.0pt, mark=o, mark options={solid, mycolor1}]
  table[row sep=crcr]{%
0.574022782359323	0\\
0.554255649047577	0.360956684965049\\
0.534460696462205	0.71075748682045\\
0.514667722609878	1.08079178535235\\
0.494874424638055	1.47013114346954\\
0.47508414387967	1.87702805446787\\
0.455291632532853	2.30130878898678\\
0.435499493393303	2.74167845859506\\
0.415707635741457	3.197318399654\\
0.395913931277966	3.66811687575492\\
0.376120170464653	4.15314055575493\\
0.356325604712792	4.65225453870157\\
0.336533155832932	5.16424664697262\\
0.316739159356293	5.69008108114092\\
0.296942936129519	6.2305648984133\\
0.277148493201861	6.78397137392247\\
0.25735387953602	7.35166140649382\\
0.237557175267672	7.93547602930466\\
0.217759758571773	8.53544378260889\\
0.197963601702039	9.15080481279147\\
0.178166646490907	9.72671690669536\\
0.15836924853432	10.3525932998672\\
0.138573213899066	11.1286574722354\\
0.118778219122921	12.1304506190317\\
0.0989818334414158	13.5591006203945\\
0.0791870355678023	15.3676292119557\\
0.0593925208757	17.7812784481005\\
0.0395964596337885	21.1609199378553\\
0.0197995584560795	26.4206011842216\\
5.81260580390308e-10	43.5703864704519\\
};
\addlegendentry{$\sigma_n = -20$ dBm}

\addplot [color=mycolor2, dashed, line width=2.0pt, mark=+, mark options={solid, mycolor2}]
  table[row sep=crcr]{%
2.45563281934323	0\\
2.37095583054819	0.58652961745376\\
2.28627883665221	1.21237195508127\\
2.20160185855494	1.86419532572767\\
2.1169249143324	2.53039523213391\\
2.03224792756009	3.20173702257439\\
1.94757107725318	3.87089124670663\\
1.8628944047229	4.53220208487679\\
1.77821753056512	5.18144556175485\\
1.69354140033674	5.81533903754865\\
1.60886477440745	6.43200692705739\\
1.5241896470999	7.02986352704788\\
1.43951371569267	7.60846142795235\\
1.35484025496592	8.16775560013486\\
1.27016578821041	8.6227819777308\\
1.18549195142568	9.17852696354813\\
1.10081878048225	9.86031793818873\\
1.01614541984246	10.68656624621\\
0.93147339126626	11.5269044861645\\
0.846800011072454	12.5124486394357\\
0.762122605158068	13.6589460322796\\
0.67744207328679	14.9317716539382\\
0.592767113496088	16.3117105126604\\
0.508089424222787	17.9050567304869\\
0.423414675824649	19.7120806380575\\
0.338753927057592	21.7706238436846\\
0.254082174751878	24.2677818076769\\
0.169400419217777	27.3940743524908\\
0.0847105524180116	31.7430703489613\\
8.46385669175259e-09	43.5721226238995\\
};
\addlegendentry{$\sigma_n = -30$ dBm}

\addplot [color=mycolor3, dotted, line width=2.0pt, mark=square, mark options={solid, mycolor3}]
  table[row sep=crcr]{%
5.47802004724716	0\\
5.28912280356319	1.08085756391362\\
5.10022556120002	2.22936691648432\\
4.91132831819603	3.38020260295893\\
4.72243107565964	4.49211745727579\\
4.53353383426511	5.54057254774717\\
4.34463659587811	6.51268400718667\\
4.15573936625203	7.40313792821053\\
3.96684217182821	8.21160804052197\\
3.77794501832707	8.81982671514256\\
3.589047927736	9.71731169219198\\
3.40015103750449	10.7303897109478\\
3.21125421298155	11.8335279273417\\
3.02235760237568	13.054973055293\\
2.83346059580165	14.3616873475762\\
2.64456417496333	15.7173531227186\\
2.45567078500938	17.101281292663\\
2.26677428211046	18.5395504474694\\
2.07787575093041	20.009070361403\\
1.88897715918314	21.5030510617551\\
1.70007922201822	23.0189340463483\\
1.51118169227964	24.5574938957904\\
1.3222844504914	26.122658478946\\
1.13338850388992	27.7234228823289\\
0.944494089058083	29.3755044536208\\
0.755603055347488	31.1058339407389\\
0.566715906727745	32.9641421601616\\
0.377838864525132	35.051563987549\\
0.188963901785062	37.6360007236239\\
8.14962617296365e-08	43.5726527292124\\
};
\addlegendentry{$\sigma_n = -40$ dBm}

\addplot [color=mycolor4, dashdotted, line width=2.0pt, mark=x, mark options={solid, mycolor4}]
  table[row sep=crcr]{%
8.76513813118726	0\\
8.46289198861462	1.67472754922688\\
8.16064584619442	3.41422167816438\\
7.85839970389637	5.0589841381668\\
7.55615356230529	6.53680384539799\\
7.25390742411744	7.82460676767851\\
6.95166129554901	8.85318854924371\\
6.64941519838104	10.0351927806525\\
6.34716912473248	11.5746291039696\\
6.04492305203519	13.2997045750327\\
5.74267725080731	15.1209792257336\\
5.44043099119824	17.0081299927157\\
5.13818469561059	18.9061267498104\\
4.83593838791141	20.7821552806459\\
4.53369215073479	22.6124414846617\\
4.23144601387905	24.3797146553672\\
3.92919988825059	26.0721632647819\\
3.62695378430843	27.6823203463566\\
3.32470769728156	29.2067930024281\\
3.02246163603244	30.6451856951401\\
2.72021560474502	31.999739953715\\
2.41796964155896	33.2751878801516\\
2.11572332341624	34.4783154098295\\
1.81347735861222	35.6180064860728\\
1.51123136073648	36.7059024040244\\
1.20898524829336	37.7582923772188\\
0.906739762333691	38.7995311331854\\
0.604500084567993	39.8740655783599\\
0.302287474078343	41.0911334625849\\
6.83909129172114e-07	43.5727486403077\\
};
\addlegendentry{$\sigma_n = -50$ dBm}

\addplot [color=mycolor5, line width=2.0pt, mark=triangle, mark options={solid, mycolor5}]
  table[row sep=crcr]{%
12.0835235447341	0\\
11.6668503189783	2.29078102934211\\
11.2501770933129	4.58327422776283\\
10.8335038678659	6.60518822459268\\
10.4168306433061	8.28317918192536\\
10.0001574217616	9.65699261275777\\
9.58348420861084	11.631225551464\\
9.16681101798249	13.9725722408206\\
8.75013777301529	16.4662012377011\\
8.33346452664863	18.9856762329431\\
7.91679128917285	21.4505644462356\\
7.50011806762715	23.8062153430022\\
7.08344484307915	26.0171485660357\\
6.66677162325038	28.0623330945188\\
6.25009840626828	29.9324248001813\\
5.83342519282216	31.6264588542926\\
5.41675196240717	33.1494971352745\\
5.00007873330747	34.5107391061931\\
4.58340552537372	35.7219601322263\\
4.16673232958544	36.7963764960548\\
3.75005916604854	37.7476154991495\\
3.3333860694612	38.5897949271126\\
2.91671313688077	39.3363543211549\\
2.50004063507021	40.0007683383104\\
2.08336919231358	40.5959759929606\\
1.66669948870797	41.1357135962601\\
1.25002678330064	41.6350607615149\\
0.833352037133684	42.1162600390945\\
0.416684570055562	42.6239683647382\\
3.19502846093212e-06	43.572867842323\\
};
\addlegendentry{$\sigma_n = -60$ dBm}

\end{axis}

\begin{axis}[%
width=4.521in,
height=1.575in,
at={(0.758in,0.481in)},
scale only axis,
xmin=0,
xlabel style={font=\color{white!15!black}},
xlabel={Per-subband rate [bps/Hz]},
ymin=0,
ylabel style={font=\color{white!15!black}},
ylabel={Power splitting ratio},
axis background/.style={fill=white},
xmajorgrids,
ymajorgrids
]
\addplot [color=mycolor1, line width=2.0pt, mark=o, mark options={solid, mycolor1}, forget plot]
  table[row sep=crcr]{%
0.574022782359323	2.22044604925031e-16\\
0.554255649047577	0.0500344144190979\\
0.534460696462205	0.0913028129225507\\
0.514667722609878	0.13189953528549\\
0.494874424638055	0.171894944482266\\
0.47508414387967	0.211225241558098\\
0.455291632532853	0.250017330191047\\
0.435499493393303	0.288238311531871\\
0.415707635741457	0.325857376178187\\
0.395913931277966	0.362940967772059\\
0.376120170464653	0.399423500906597\\
0.356325604712792	0.435360543681527\\
0.336533155832932	0.470732584347999\\
0.316739159356293	0.505600241286251\\
0.296942936129519	0.539871188843706\\
0.277148493201861	0.573659367613965\\
0.25735387953602	0.606984677165034\\
0.237557175267672	0.639771760028363\\
0.217759758571773	0.67211459990059\\
0.197963601702039	0.704035217671119\\
0.178166646490907	0.730490309730776\\
0.15836924853432	0.750531639449964\\
0.138573213899066	0.762308439519426\\
0.118778219122921	0.769628912723413\\
0.0989818334414158	0.780509252665819\\
0.0791870355678023	0.791710248323641\\
0.0593925208757	0.806195539201567\\
0.0395964596337885	0.831027418342949\\
0.0197995584560795	0.87628461076199\\
5.81260580390308e-10	0.999982386565607\\
};
\addplot [color=mycolor2, dashed, line width=2.0pt, mark=+, mark options={solid, mycolor2}, forget plot]
  table[row sep=crcr]{%
2.45563281934323	2.22044604925031e-16\\
2.37095583054819	0.0704382566111917\\
2.28627883665221	0.136344583941512\\
2.20160185855494	0.198292964520753\\
2.1169249143324	0.256524132111811\\
2.03224792756009	0.311265535886801\\
1.94757107725318	0.362724267022017\\
1.8628944047229	0.411091248813662\\
1.77821753056512	0.456540898905806\\
1.69354140033674	0.499256924773358\\
1.60886477440745	0.539365671919436\\
1.5241896470999	0.577062612693091\\
1.43951371569267	0.612410706406503\\
1.35484025496592	0.645646281789774\\
1.27016578821041	0.666682140325566\\
1.18549195142568	0.684456961026731\\
1.10081878048225	0.700039270678062\\
1.01614541984246	0.720881884940469\\
0.93147339126626	0.73022085490169\\
0.846800011072454	0.736951711635312\\
0.762122605158068	0.753431282541991\\
0.67744207328679	0.771556799356008\\
0.592767113496088	0.77918946799338\\
0.508089424222787	0.797118284638796\\
0.423414675824649	0.817727367861008\\
0.338753927057592	0.82314560719201\\
0.254082174751878	0.845590761565754\\
0.169400419217777	0.873758719129337\\
0.0847105524180116	0.910908860956578\\
8.46385669175259e-09	0.999979654792071\\
};
\addplot [color=mycolor3, dotted, line width=2.0pt, mark=square, mark options={solid, mycolor3}, forget plot]
  table[row sep=crcr]{%
5.47802004724716	2.22044604925031e-16\\
5.28912280356319	0.125584579280922\\
5.10022556120002	0.235360080915002\\
4.91132831819603	0.331319602319782\\
4.72243107565964	0.415185833659928\\
4.53353383426511	0.488460122216869\\
4.34463659587811	0.552450854824023\\
4.15573936625203	0.608302959070357\\
3.96684217182821	0.656954628615161\\
3.77794501832707	0.680169827379972\\
3.589047927736	0.701997798204197\\
3.40015103750449	0.719607750073445\\
3.21125421298155	0.726603038585099\\
3.02235760237568	0.738827833888948\\
2.83346059580165	0.753740291600599\\
2.64456417496333	0.766790068656538\\
2.45567078500938	0.773864571686622\\
2.26677428211046	0.786203144454398\\
2.07787575093041	0.799875101905173\\
1.88897715918314	0.813757760158667\\
1.70007922201822	0.827726449676601\\
1.51118169227964	0.84176345159803\\
1.3222844504914	0.855926836431421\\
1.13338850388992	0.870234390070171\\
0.944494089058083	0.884810642129082\\
0.755603055347488	0.899869220451413\\
0.566715906727745	0.91578239034175\\
0.377838864525132	0.933303345600705\\
0.188963901785062	0.954411807514714\\
8.14962617296365e-08	0.999979405855629\\
};
\addplot [color=mycolor4, dashdotted, line width=2.0pt, mark=x, mark options={solid, mycolor4}, forget plot]
  table[row sep=crcr]{%
8.76513813118726	2.22044604925031e-16\\
8.46289198861462	0.189185435828981\\
8.16064584619442	0.34204258514641\\
7.85839970389637	0.465452892271423\\
7.55615356230529	0.564958832110637\\
7.25390742411744	0.645005923287428\\
6.95166129554901	0.698612855670175\\
6.64941519838104	0.716846705215501\\
6.34716912473248	0.730735066794488\\
6.04492305203519	0.745961431222048\\
5.74267725080731	0.758033769730055\\
5.44043099119824	0.774825282873592\\
5.13818469561059	0.792757529971644\\
4.83593838791141	0.810476507788091\\
4.53369215073479	0.827368466075844\\
4.23144601387905	0.843278990805401\\
3.92919988825059	0.858170463286767\\
3.62695378430843	0.872106186396769\\
3.32470769728156	0.885062586515748\\
3.02246163603244	0.897093159401537\\
2.72021560474502	0.9082659056767\\
2.41796964155896	0.918665219418154\\
2.11572332341624	0.92837033353614\\
1.81347735861222	0.937507288579668\\
1.51123136073648	0.946181521765273\\
1.20898524829336	0.954540162185645\\
0.906739762333691	0.962791032248647\\
0.604500084567993	0.971288996310122\\
0.302287474078343	0.980877467789495\\
6.83909129172114e-07	0.999980842155314\\
};
\addplot [color=mycolor5, line width=2.0pt, mark=triangle, mark options={solid, mycolor5}, forget plot]
  table[row sep=crcr]{%
12.0835235447341	2.22044604925031e-16\\
11.6668503189783	0.250473026052327\\
11.2501770933129	0.43725237485522\\
10.8335038678659	0.576198289578071\\
10.4168306433061	0.679104276060216\\
10.0001574217616	0.721244018992092\\
9.58348420861084	0.733167229838322\\
9.16681101798249	0.749833839209567\\
8.75013777301529	0.772703256940026\\
8.33346452664863	0.797228167214537\\
7.91679128917285	0.820650496433475\\
7.50011806762715	0.842156763734519\\
7.08344484307915	0.86159724109489\\
6.66677162325038	0.87904145594278\\
6.25009840626828	0.894534350024368\\
5.83342519282216	0.908225912629898\\
5.41675196240717	0.92028340530386\\
5.00007873330747	0.930888698632061\\
4.58340552537372	0.940211537540374\\
4.16673232958544	0.948404480573165\\
3.75005916604854	0.955612272718496\\
3.3333860694612	0.961967880311013\\
2.91671313688077	0.96759306913173\\
2.50004063507021	0.972601558786326\\
2.08336919231358	0.977100771119742\\
1.66669948870797	0.98119790776949\\
1.25002678330064	0.985012597909313\\
0.833352037133684	0.988717304799494\\
0.416684570055562	0.99265719240821\\
3.19502846093212e-06	0.999985971731636\\
};
\end{axis}
\end{tikzpicture}%
					}
				}
				\subfloat[AP-IRS horizontal distance\label{fi:re_distance}]{
					\resizebox{0.45\columnwidth}{!}{
						% This file was created by matlab2tikz.
%
%The latest updates can be retrieved from
%  http://www.mathworks.com/matlabcentral/fileexchange/22022-matlab2tikz-matlab2tikz
%where you can also make suggestions and rate matlab2tikz.
%
\definecolor{mycolor1}{rgb}{0.00000,0.44700,0.74100}%
\definecolor{mycolor2}{rgb}{0.85000,0.32500,0.09800}%
\definecolor{mycolor3}{rgb}{0.92900,0.69400,0.12500}%
\definecolor{mycolor4}{rgb}{0.49400,0.18400,0.55600}%
%
\begin{tikzpicture}

\begin{axis}[%
width=4.521in,
height=3.566in,
at={(0.758in,0.481in)},
scale only axis,
xmin=0,
xlabel style={font=\color{white!15!black}},
xlabel={Per-subband rate [bps/Hz]},
ymin=0,
ylabel style={font=\color{white!15!black}},
ylabel={Average output DC current [$\mu$A]},
axis background/.style={fill=white},
xmajorgrids,
ymajorgrids,
legend style={legend cell align=left, align=left, draw=white!15!black}
]
\addplot [color=mycolor1, line width=2.0pt, mark=o, mark options={solid, mycolor1}]
  table[row sep=crcr]{%
5.49292547008313	0\\
5.30351424656718	1.22860642710915\\
5.1141030238189	2.57951019637698\\
4.92469180093791	3.95479925295618\\
4.73528057893268	5.29349826890822\\
4.54586936161079	6.55983783612091\\
4.35645815422699	7.73499704089284\\
4.16704695697714	8.77804045088843\\
3.97763581113116	9.74959289942188\\
3.78822475574737	10.799818731065\\
3.59881373998083	11.9921160700183\\
3.40940307106706	13.391163391371\\
3.21999269800457	14.9283371609607\\
3.03058244437993	16.5600196515875\\
2.841172703132	18.2734756225963\\
2.65176228520283	20.0581600890651\\
2.46235248777006	21.88873631099\\
2.27294226376601	23.752937131981\\
2.08353141887978	25.6422244008408\\
1.89412169249148	27.5479021784175\\
1.70471205536012	29.4662928897233\\
1.51530288906667	31.397694670762\\
1.32589360617063	33.3501360214671\\
1.13648588766029	35.336188145785\\
0.947079190469388	37.3719834596829\\
0.757674283559347	39.4889495791428\\
0.568271349769703	41.7439549566449\\
0.378873215899665	44.2558933600272\\
0.189470889946183	47.3370315644049\\
1.21979687880433e-06	54.295030579393\\
};
\addlegendentry{$d_H = 2$}

\addplot [color=mycolor2, dashed, line width=2.0pt, mark=+, mark options={solid, mycolor2}]
  table[row sep=crcr]{%
5.33789505741759	0\\
5.15382971020342	1.09941596335751\\
4.96976436389307	2.29718251172375\\
4.78569901875289	3.51164207596126\\
4.60163367383018	4.69184722884272\\
4.41756833996081	5.80791623076107\\
4.23350304173724	6.84412491808461\\
4.04943781785939	7.77803031109085\\
3.86537260823593	8.63514550617198\\
3.68130772958912	9.53054320949779\\
3.4972432495151	10.5476276750572\\
3.31317843568965	11.7263278158264\\
3.12911419227665	13.04532335478\\
2.94505014456926	14.446694963118\\
2.7609863023879	15.9122132567759\\
2.57692280904887	17.4385670954059\\
2.3928586924957	19.0126596309592\\
2.20879463989344	20.6187539048144\\
2.02473125559986	22.2484341568247\\
1.84066826390564	23.8952162932067\\
1.65660389671081	25.5586419375961\\
1.47254223600832	27.2364911599142\\
1.28847885495068	28.935597767701\\
1.10441734079397	30.6608869511376\\
0.920355936740993	32.4336298365762\\
0.736297734562477	34.282440828552\\
0.552241626444198	36.2565644630476\\
0.368187386690097	38.4605898246309\\
0.184129866018214	41.1687613059758\\
1.15569393126814e-06	47.2967566604214\\
};
\addlegendentry{$d_H = 4$}

\addplot [color=mycolor3, dotted, line width=2.0pt, mark=square, mark options={solid, mycolor3}]
  table[row sep=crcr]{%
5.28039296574179	0\\
5.09831044954266	1.05592969342657\\
4.9162279340564	2.20265514507454\\
4.73414542153245	3.36371487165505\\
4.55206290883948	4.4913761658927\\
4.36998041814651	5.55763782162234\\
4.18789799374451	6.54778083866998\\
4.00581572207433	7.45595158789979\\
3.82373346997206	8.26239057254163\\
3.64165168363283	9.11056689151205\\
3.45957002004607	10.0754547099623\\
3.27748802506296	11.1808034640228\\
3.09540657986697	12.4263381844038\\
2.91332552301789	13.7548080756097\\
2.7312442851038	15.1418035148798\\
2.5491635757865	16.5816047792508\\
2.36708314086048	18.0725172055314\\
2.18500260029932	19.5933300550087\\
2.00292169320697	21.1380080973959\\
1.82084211072734	22.7008109898492\\
1.63876154240272	24.2784732345466\\
1.45668275220849	25.8721460466841\\
1.27460320776707	27.4864490901215\\
1.09252345855059	29.1288500924282\\
0.91044555777666	30.8135816862508\\
0.72836968683994	32.5745564550693\\
0.546294520347069	34.4567616718868\\
0.364223529048754	36.5588750151113\\
0.18214895188615	39.1427080559914\\
2.35268032435636e-06	44.9941851314238\\
};
\addlegendentry{$d_H = 6$}

\addplot [color=mycolor4, dashdotted, line width=2.0pt, mark=x, mark options={solid, mycolor4}]
  table[row sep=crcr]{%
5.25729943457622	0\\
5.07601324733798	1.03908816950273\\
4.89472706054071	2.16611496068299\\
4.71344087820199	3.30659619160555\\
4.53215471428814	4.41402199151754\\
4.35086853985646	5.46109871951431\\
4.1695824800003	6.43350677644179\\
3.9882966376107	7.32553288057761\\
3.80701076487798	8.12588727042968\\
3.62572537870195	8.94584329664644\\
3.44443988803284	9.89135822332428\\
3.26315424854803	10.9749984323839\\
3.08186919532869	12.1883717407968\\
2.90058480850501	13.4865779474332\\
2.71929971735184	14.8447834207999\\
2.53801572080592	16.2546188808448\\
2.35673144393694	17.7112033742202\\
2.1754469892468	19.1999656478683\\
1.99416330576355	20.7120435290636\\
1.81288028303869	22.2419159006322\\
1.63159636627985	23.7873410717587\\
1.45031235781618	25.3497801877808\\
1.26902990147138	26.9315774481146\\
1.08774632623237	28.5428322257801\\
0.906464123666369	30.1960132143176\\
0.72518601917496	31.9201395832939\\
0.543908307120818	33.7660613015928\\
0.362632898832289	35.8283900996869\\
0.181353474703628	38.3647275801675\\
2.29536213695632e-06	44.1096574359739\\
};
\addlegendentry{$d_H = 7.5$}

\end{axis}
\end{tikzpicture}%
					}
				}
				\\
				\subfloat[Transmit antennas\label{fi:re_tx}]{
					\resizebox{0.45\columnwidth}{!}{
						% This file was created by matlab2tikz.
%
%The latest updates can be retrieved from
%  http://www.mathworks.com/matlabcentral/fileexchange/22022-matlab2tikz-matlab2tikz
%where you can also make suggestions and rate matlab2tikz.
%
\definecolor{mycolor1}{rgb}{0.00000,0.44706,0.74118}%
\definecolor{mycolor2}{rgb}{0.85098,0.32549,0.09804}%
\definecolor{mycolor3}{rgb}{0.92941,0.69412,0.12549}%
%
\begin{tikzpicture}

\begin{axis}[%
width=4.036in,
height=3.396in,
at={(0.677in,0.458in)},
scale only axis,
xmin=0,
xlabel style={font=\color{white!15!black}},
xlabel={Per-subband rate [bps/Hz]},
ymin=0,
ylabel style={font=\color{white!15!black}},
ylabel={Output DC current [$\mu$A]},
axis background/.style={fill=white},
xmajorgrids,
ymajorgrids,
legend style={legend cell align=left, align=left, draw=white!15!black},
title style={font=\huge}, label style={font=\huge}, ticklabel style={font=\LARGE}, legend style={font=\LARGE}
]
\addplot [color=mycolor1, line width=2.0pt, mark=o, mark options={solid, mycolor1}]
  table[row sep=crcr]{%
7.89879461020509	2.22044604925031e-10\\
7.48306857796699	20.5884848741448\\
7.06734254593788	50.0968598730126\\
6.65161651388302	79.6493973029652\\
6.23589048201876	104.533985242594\\
5.82016445270511	139.189556420904\\
5.4044384242978	187.135713002996\\
4.9887123893292	241.231302874683\\
4.57298635368985	298.608843204122\\
4.1572603212796	357.141357284625\\
3.7415342892566	415.21185994383\\
3.32580825719106	471.669166663569\\
2.91008222529047	525.792584067963\\
2.4943561934017	577.245593603288\\
2.07863016220111	626.043118645554\\
1.66290414735944	672.550594074983\\
1.2471783610897	717.583259853349\\
0.831453520769062	762.75961634867\\
0.415732690959845	812.310785101648\\
5.95987027451057e-10	910.736770067952\\
};
\addlegendentry{BCD: $M = 1$}

\addplot [color=mycolor1, dashed, line width=2.0pt, mark=+, mark options={solid, mycolor1}]
  table[row sep=crcr]{%
7.89879461020783	2.22044604925031e-10\\
7.74333304378191	2.34206118460941\\
7.57927798611362	5.63582781443343\\
7.40539581003889	10.0745453118942\\
7.22043982842399	15.9783805758471\\
7.02292277608993	23.7943956853936\\
6.81103008998867	34.096540302997\\
6.58253192848948	47.5856214826701\\
6.33463897717353	65.0893338682385\\
6.06380398978433	87.5622351399993\\
5.7654317542847	116.085763234144\\
5.43342965164188	151.868232260292\\
5.05950603532979	196.244846002231\\
4.63202961019445	250.677730230468\\
4.13411896237966	316.755847522271\\
3.54040507205419	396.195115257199\\
2.81199338522465	490.838304310137\\
1.89368025438118	602.655095002329\\
0.767905709053224	733.74206924775\\
0	886.322598824754\\
};
\addlegendentry{LC-BCD: $M = 1$}

\addplot [color=mycolor2, line width=2.0pt, mark=square, mark options={solid, mycolor2}]
  table[row sep=crcr]{%
8.60500212376507	2.22044604925031e-10\\
8.15210727503012	47.4428629176451\\
7.6992124264879	120.377428625059\\
7.24631757791559	192.936041989418\\
6.79342272944819	256.155175086373\\
6.3405278814746	364.388285366371\\
5.88763303263547	498.171980652814\\
5.43473818349643	644.876330474064\\
4.98184333487081	797.29885958874\\
4.52894848630435	949.737701534857\\
4.07605363775018	1097.98836818163\\
3.62315878922321	1239.25297140661\\
3.17026394092251	1371.96022703329\\
2.71736909238362	1495.56003701423\\
2.26447424380159	1610.3647410394\\
1.81157939532429	1717.47943055802\\
1.35868454737594	1818.95850509955\\
0.905789764599466	1918.52345985021\\
0.452896605221121	2025.12408888762\\
1.38344061614948e-10	2231.70337515085\\
};
\addlegendentry{BCD: $M = 2$}

\addplot [color=mycolor2, dashed, line width=2.0pt, mark=x, mark options={solid, mycolor2}]
  table[row sep=crcr]{%
8.60500212414857	2.22044604925031e-10\\
8.44938650233237	4.14113484894377\\
8.28499050422935	10.6459605624202\\
8.1106997778157	19.9925206224927\\
7.92525106041057	32.9749414756821\\
7.727124334841	50.7034202152108\\
7.51447165677215	74.604182408811\\
7.28500577476184	106.41949708063\\
7.03586150063108	148.207657985464\\
6.76337773267469	202.342996045141\\
6.46278000351083	271.515881640725\\
6.12768471639755	358.732727013173\\
5.74930067706576	467.315980543692\\
5.31508190501384	600.904188116143\\
4.80633530745484	763.451916271768\\
4.19372854984657	959.229768020339\\
3.42853860322894	1192.82443440514\\
2.42728800564053	1469.13865403107\\
1.08588673410311	1793.39123660647\\
0	2171.11696187099\\
};
\addlegendentry{LC-BCD: $M = 2$}

\addplot [color=mycolor3, line width=2.0pt, mark=triangle, mark options={solid, mycolor3}]
  table[row sep=crcr]{%
9.34693483985911	2.22044604925031e-10\\
8.85499090077297	112.023244878406\\
8.36304696194518	292.402251097562\\
7.87110302305593	469.909034581426\\
7.37915908416736	644.009055673124\\
6.88721514518741	948.22589806267\\
6.39527120623294	1304.04695855599\\
5.90332726731093	1687.33417336749\\
5.41138332845874	2078.84593641839\\
4.91943938955505	2463.75070252656\\
4.42749545067345	2831.66387846562\\
3.93555151196377	3176.18083506754\\
3.4436075731816	3494.18294374424\\
2.95166363435452	3785.13866703171\\
2.45971969541455	4050.54748153993\\
1.96777575657887	4293.64583533187\\
1.47583181771894	4519.61087758234\\
0.983887880182558	4737.02997338988\\
0.491944276655667	4965.04697185014\\
1.71787113686514e-11	5396.15063996105\\
};
\addlegendentry{BCD: $M = 4$}

\addplot [color=mycolor3, dashed, line width=2.0pt, mark=triangle, mark options={solid, rotate=180, mycolor3}]
  table[row sep=crcr]{%
9.346921216299	2.22044604925031e-10\\
9.19115494401357	7.59936349992529\\
9.02653807429387	20.9665005718635\\
8.85198104954077	41.2646293679712\\
8.66620900588006	70.4283496776819\\
8.46768524362875	111.163618753492\\
8.25453587917671	166.947727517477\\
8.02444217285122	242.029303857007\\
7.77448829908568	341.428299998728\\
7.50093900788716	470.935991598839\\
7.19890264005329	637.114997349465\\
6.86180451533609	847.301464118626\\
6.48051854487751	1109.59708262551\\
6.04188014329446	1432.88014134287\\
5.52594248932261	1826.79865172316\\
4.90055322844244	2301.81359566659\\
4.109458318449	2869.0427107228\\
3.04490618272372	3540.48127055286\\
1.50395123292928	4328.86298561351\\
0	5247.69275114493\\
};
\addlegendentry{LC-BCD: $M = 4$}

\end{axis}
\end{tikzpicture}%

					}
				}
				\subfloat[Reflecting elements\label{fi:re_reflector}]{
					\resizebox{0.45\columnwidth}{!}{
						% This file was created by matlab2tikz.
%
%The latest updates can be retrieved from
%  http://www.mathworks.com/matlabcentral/fileexchange/22022-matlab2tikz-matlab2tikz
%where you can also make suggestions and rate matlab2tikz.
%
\definecolor{mycolor1}{rgb}{0.00000,0.44700,0.74100}%
\definecolor{mycolor2}{rgb}{0.85000,0.32500,0.09800}%
\definecolor{mycolor3}{rgb}{0.92900,0.69400,0.12500}%
\definecolor{mycolor4}{rgb}{0.49400,0.18400,0.55600}%
\definecolor{mycolor5}{rgb}{0.46600,0.67400,0.18800}%
\definecolor{mycolor6}{rgb}{0.30100,0.74500,0.93300}%
\definecolor{mycolor7}{rgb}{0.63500,0.07800,0.18400}%
%
\begin{tikzpicture}

\begin{axis}[%
width=4.521in,
height=3.566in,
at={(0.758in,0.481in)},
scale only axis,
xmin=0,
xlabel style={font=\color{white!15!black}},
xlabel={Per-subband rate [bps/Hz]},
ymin=0,
ylabel style={font=\color{white!15!black}},
ylabel={Average output DC current [$\mu$A]},
axis background/.style={fill=white},
xmajorgrids,
ymajorgrids,
legend style={legend cell align=left, align=left, draw=white!15!black}
]
\addplot [color=mycolor1, line width=2.0pt, mark=o, mark options={solid, mycolor1}]
  table[row sep=crcr]{%
4.34205605582246	0\\
4.19233624508348	0.547806545574378\\
4.04261064239603	1.11106985990734\\
3.89288498980317	1.66768096923848\\
3.74315810146432	2.20282578723515\\
3.59343156201191	2.7077853322161\\
3.44370527106509	3.17820143037213\\
3.29397985154681	3.6124144105547\\
3.14425432035022	3.9998539341622\\
2.99453052768507	4.35839846793078\\
2.8448065945008	4.73564184457831\\
2.69508497420668	5.1692032708104\\
2.54536397944092	5.60897765654803\\
2.39564361514834	6.10212065003164\\
2.24592504654141	6.63562423886325\\
2.09620656356437	7.20514951811736\\
1.94649023127123	7.80181649069291\\
1.79677434164088	8.41299691140024\\
1.64705845169537	9.04260421680787\\
1.49734181424063	9.68643733053296\\
1.34762661134595	10.3429862528904\\
1.1979062100068	11.0139576896306\\
1.04818705147234	11.6979504156107\\
0.89846688820435	12.404672653144\\
0.748742796433798	13.1368848055671\\
0.599017405387487	13.9137015608961\\
0.449284893267714	14.7550158254777\\
0.299555923253412	15.7066250964484\\
0.149816921936368	16.89047888913\\
7.06138972267772e-06	19.6036121220506\\
};
\addlegendentry{$M = 4, L = 0$}

\addplot [color=mycolor2, dashed, line width=2.0pt, mark=+, mark options={solid, mycolor2}]
  table[row sep=crcr]{%
5.28309036011037	0\\
5.10091511108678	1.17353653866223\\
4.91873972203469	2.4817791827314\\
4.73656447106627	3.82086622893429\\
4.5543894601547	5.12635385647139\\
4.37221390622527	6.36066112187699\\
4.1900385642305	7.50406592496986\\
4.00786324356599	8.49427079489113\\
3.8256884336478	9.49808118929626\\
3.64351349310535	10.5949727597888\\
3.46133911314663	11.8541474120544\\
3.27916512297228	13.2733186883564\\
3.09699016960904	14.832991528707\\
2.91481679629074	16.4763342225571\\
2.73264429159564	18.1980910314739\\
2.55047189336053	19.9834465646251\\
2.36829879759566	21.8077103915807\\
2.18612661015421	23.6631519207784\\
2.00395267439694	25.5364323989953\\
1.82177557663813	27.4231682387703\\
1.63960275663747	29.3170441086272\\
1.45742695559613	31.2208955707869\\
1.27525229277959	33.1377769844004\\
1.09307783647839	35.0794506806848\\
0.910903715495796	37.0634495211419\\
0.728734186459102	39.1208444574805\\
0.546567293691612	41.307972410895\\
0.364402939974913	43.7382966992671\\
0.182235561214653	46.7087466301273\\
3.47544201686259e-06	53.4101666321577\\
};
\addlegendentry{$M = 4, L = 10$}

\addplot [color=mycolor3, dotted, line width=2.0pt, mark=square, mark options={solid, mycolor3}]
  table[row sep=crcr]{%
5.47618129878899	0\\
5.28734746037613	1.31959742796363\\
5.09851362311337	2.81074181141752\\
4.90967978568931	4.34382736634757\\
4.72084595213547	5.83925300799064\\
4.53201212736014	7.25091152733123\\
4.34317834264694	8.55488263474232\\
4.15434461020069	9.68043249537239\\
3.96551103977712	10.9402778038053\\
3.77667770737339	12.2082084225519\\
3.58784477107584	13.732096889931\\
3.39901209086199	15.4252432909447\\
3.21017954609396	17.284272776853\\
3.02134634320763	19.2497171623413\\
2.83251373668926	21.2971587715387\\
2.64368106583542	23.4117525451575\\
2.45484972798562	25.5644241329951\\
2.26601887075218	27.7404989449924\\
2.077186947687	29.9348920987101\\
1.88835394022212	32.1362185154834\\
1.69952181526151	34.3405209334158\\
1.51068968093638	36.5460647710763\\
1.32185779986247	38.7647201473236\\
1.13302747878649	41.0025476615823\\
0.94419534477912	43.285654085791\\
0.755366774196025	45.6459102949649\\
0.566539822937346	48.1485734500858\\
0.377715808623576	50.921035028057\\
0.188888570729196	54.3016402251254\\
3.6248963084413e-06	61.9536811183472\\
};
\addlegendentry{$M = 4, L = 20$}

\addplot [color=mycolor4, dashdotted, line width=2.0pt, mark=x, mark options={solid, mycolor4}]
  table[row sep=crcr]{%
6.47486216709289	0\\
6.25159105752932	2.98109398975802\\
6.02831994857846	6.65306772454862\\
5.80504883942132	10.5579689566499\\
5.58177773028909	14.4176956722895\\
5.35850662150384	18.0727643168127\\
5.13523551350737	21.4405962193126\\
4.91196440773918	24.1803539920299\\
4.68869330710523	27.6156189428602\\
4.46542221627301	31.9504007192536\\
4.24215115648416	36.8763238401629\\
4.01888011757192	42.3869756263511\\
3.79560901881831	48.2536527379305\\
3.57233799363469	54.3610656944757\\
3.34906693186252	60.6365823593092\\
3.12579578841	67.0131568021281\\
2.90252457452086	73.434192217287\\
2.67925340160483	79.85444590983\\
2.45598247758194	86.2441647701629\\
2.23271156441007	92.5748347224874\\
2.00944011419755	98.839000943652\\
1.78616901136754	105.037725458468\\
1.5628983114495	111.185029852466\\
1.33962772736202	117.310463970055\\
1.11635747010917	123.466475435595\\
0.89308812132432	129.73964627946\\
0.669820638651518	136.291405282531\\
0.446557191151867	143.438566028914\\
0.223298118314556	152.02076571762\\
1.76777994519408e-08	171.011001692578\\
};
\addlegendentry{$M = 4, L = 40$}

\addplot [color=mycolor5, line width=2.0pt, mark=triangle, mark options={solid, mycolor5}]
  table[row sep=crcr]{%
7.51350104773273	0\\
7.25441480439253	7.10392884864182\\
6.99532856163041	16.8368515875128\\
6.73624231860965	27.5220438532997\\
6.47715607562188	38.1664544995199\\
6.21806983264587	48.2160569942795\\
5.95898358974191	57.2145395807764\\
5.69989734729285	65.4788894469207\\
5.44081110609439	77.6348765187148\\
5.18172486506217	92.8340596521314\\
4.92263862187084	109.759412808993\\
4.66355237527503	127.847564611429\\
4.40446613158631	146.723542127331\\
4.14537988847428	166.101980868642\\
3.88629364547878	185.73714487125\\
3.62720740251815	205.421834241243\\
3.36812115951283	224.987123938573\\
3.10903491654381	244.300776843303\\
2.84994867355738	263.267393842955\\
2.59086243072511	281.824361551447\\
2.33177618783302	299.945450092923\\
2.07268994502637	317.633313923235\\
1.8136037020451	334.933644543756\\
1.5545174593051	351.926961856166\\
1.29543122219541	368.758692791601\\
1.03634506905386	385.659032651519\\
0.777259359096223	403.025987286049\\
0.518175314962515	421.650495001314\\
0.259094919463662	443.617573418971\\
7.42892383281577e-10	491.367101394242\\
};
\addlegendentry{$M = 4, L = 80$}

\addplot [color=mycolor6, dashed, line width=2.0pt, mark=triangle, mark options={solid, rotate=180, mycolor6}]
  table[row sep=crcr]{%
8.25283288157835	0\\
7.96825243714103	15.8852633234396\\
7.68367199311573	40.3178601497957\\
7.39909154896587	67.8011034797296\\
7.11451110479976	95.2024051264442\\
6.82993066062878	120.853397603828\\
6.54535021653568	142.587541539228\\
6.26076977271062	171.936109004477\\
5.97618932835029	213.149032380614\\
5.69160888388861	259.739104872306\\
5.40702843968808	309.910847088957\\
5.12244799549759	362.557502514059\\
4.83786755131617	416.70640541676\\
4.55328710715761	471.511800588204\\
4.26870666298166	526.259033194246\\
3.98412621881416	580.364729968211\\
3.69954577468746	633.372544593642\\
3.41496533049704	684.945246946289\\
3.13038488645245	734.854837290194\\
2.84580444250682	782.973939573082\\
2.56122399842932	829.265676247766\\
2.2766435542361	873.781806523876\\
1.99206311011748	916.659250765633\\
1.70748266589581	958.139446679172\\
1.42290222177333	998.578196997582\\
1.1383217781099	1038.53915146025\\
0.853741349107719	1078.9313793402\\
0.569161204424299	1121.50896277668\\
0.284584161349985	1170.81861474896\\
5.52815908899273e-11	1275.81358725888\\
};
\addlegendentry{$M = 4, L = 120$}

\addplot [color=mycolor7, dotted, line width=2.0pt, mark=o, mark options={solid, mycolor7}]
  table[row sep=crcr]{%
8.85859004857782	0\\
8.55312142575715	29.2971611290716\\
8.24765280379629	77.3607143403392\\
7.94218418145227	132.004623507413\\
7.63671555911429	186.421726128288\\
7.33124693676512	237.079561686269\\
7.02577831450788	276.941404756477\\
6.72030969204934	350.604379080774\\
6.41484106968256	440.392196253696\\
6.10937244732108	539.588005379397\\
5.80390382494415	645.53193369428\\
5.49843520259238	755.867787662412\\
5.19296658025684	868.509009830119\\
4.88749795795566	981.659698359078\\
4.58202933558859	1093.82794791071\\
4.27656071328778	1203.82233102053\\
3.9710920909508	1310.73629411772\\
3.66562346866408	1413.92431303439\\
3.36015484652549	1512.97522446779\\
3.05468622435413	1607.68577578655\\
2.74921760207527	1698.03860089322\\
2.4437489796021	1784.18685952716\\
2.13828035721773	1866.44654271617\\
1.8328117349554	1945.31418536681\\
1.52734311261914	2021.52079744522\\
1.22187449046473	2096.11516036061\\
0.916405869010911	2170.79977698504\\
0.610937309115371	2248.72293073253\\
0.305470431523558	2337.9879973321\\
2.03433583657618e-11	2525.38262999629\\
};
\addlegendentry{$M = 4, L = 160$}

\end{axis}
\end{tikzpicture}%
					}
				}
				\caption{Average achievable R-E region under different configurations.}
			\end{figure}
			\label{re:2.7}
		\end{response}

		\begin{point}
			How practical is it to consider lossless reflection from the RIS? Specifically, by considering the magnitude to be 1, the reflection losses at the RIS have been ignored.
			\label{pt:2.8}
		\end{point}

		\begin{response}
			Green's decomposition suggests that the backscattered/reflected signal of an antenna can be decomposed into the \emph{structural mode} component and the \emph{antenna mode} component. The former is fixed for a given antenna and can be regarded as part of the environment multipath, while the latter is adjustable and depends on the mismatch of the antenna and load impedances. The reflection coefficient of the $l$-th IRS element is thus defined as
			\begin{equation}
				\phi_l = \frac{Z_l - Z_0}{Z_l + Z_0}
			\end{equation}
			where $Z_0$ is the characteristic impedance and $Z_l = R_l + j X_l$ is the impedance of the $l$-th IRS element. It implies that $\lvert{\phi_l}\rvert \le 1$ for $R_l \ge 0$, and we assume $Z_l$ is pure reactive such that
			\begin{equation}
				\phi_l = \frac{j X_l - Z_0}{j X_l + Z_0} = e^{j \theta_{l}}
			\end{equation}
			which corresponds to the lossless reflection model adopted in the paper. Nevertheless, due to the non-zero power consumption at the IRS in practice, $R_l$ would be positive and the reflection coefficient $\phi_l$ not only has a magnitude smaller than 1 but also depends on the phase shift. [23] found that $\lvert{\phi}_l\rvert$ experiences a trough as low as \num{0.2} for $R_l = 2.5\,\si{\ohm}$ at zero phase shift, which suggests that the position and direction of the IRS should be carefully designed.
			\label{re:2.8}
		\end{response}

		\begin{point}
			Minor comment: The size of all the numerical results figures is too small.
			\label{pt:2.9}
		\end{point}

		\begin{response}
			We sincerely apologize for the inconvenience. The figures would be amplified in future versions.
			\label{re:2.9}
		\end{response}
	\end{reviewersection}

	\begin{reviewersection}
		\begin{point}
			First of all, motivations of studying the IRS on SWIPT is very unclear to me. Please clarify.
			\label{pt:3.1}
		\end{point}

		\begin{response}
			We appreciate your suggestion and have modified the manuscript to emphasize this point. A major challenge for SWIPT is that the information decoder and energy harvester have different sensitivity -- although conventional radios as Wi-Fi and Bluetooth can support a signal strength of \num{-90} to \SI{-100}{\dBm}, most existing harvesters only capture signals at \num{-20} to \SI{-30}{\dBm}. Since the transmit power is strictly subject to regulations, it is crucial to increase the energy efficiency to boost the signal strength and extend the system coverage. Therefore, we believe the effective channel enhancement and the low power consumption of IRS is a perfect match for SWIPT.
			\label{re:3.1}
		\end{response}

		\begin{point}
			Also, the contributions of this work are rather unclear, and thus, those should be better mentioned.
			\label{pt:3.2}
		\end{point}

		\begin{response}
			Thank you for the opinion and the manuscript has been revised accordingly. This paper proposed an IRS-aided SWIPT network and considered a joint waveform and beamforming design over spatial and frequency domains under energy harvester nonlinearity. To the best of our knowledge, relevant existing papers assumed linear harvester model where the RF-to-DC conversion efficiency is a constant regardless of its input. However, the harvester consists of an antenna to intercept RF signal and a diode rectifier (nonlinear device) to produce DC power, and the RF-to-DC efficiency indeed depends on the power and shape of the input waveform. By exploiting the joint impact of passive beamforming and harvester nonlinearity, we found that:
			\begin{itemize}
				\item the IRS should be developed next to the transmitter or the receiver due to double fading;
				\item the optimal transceiving mode depends on the SNR and the number of subbands;
				\item the power scaling laws of active and passive beamforming are in the order of $M^2$ and $L^4$, respectively;
				\item the IRS reflection coefficient cannot be designed per subband and there exists a tradeoff in subchannel enhancement;
				\item the optimal passive beamforming can be approximated in closed form for narrowband transmission but varies at different R-E points for broadband transmission.
			\end{itemize}
			\label{re:3.2}
		\end{response}

		\begin{point}
			Please explain the derived results more intuitively for better understanding.
			\label{pt:3.3}
		\end{point}

		\begin{response}
			TODO
			\label{re:3.3}
		\end{response}

		\begin{point}
			Authors assumed the unrealistic situation: the channels are assumed to be perfectly known. However, in practice, the channel should be estimated, e.g., as studied in \cite{Kang2020}, \cite{You2019}. It would be much better to discuss the channel estimation issue by citing the above references.
			\label{pt:3.4}
		\end{point}

		\begin{response}
			The reviewer is referred to Response \ref{re:1.1}. We have incorporated the channel estimation issue into the revised manuscript by citing those references.
			\label{re:3.4}
		\end{response}

		\begin{point}
			More simulation results should be added to better and aggregately validate the effectiveness of the proposed method.
			\label{pt:3.5}
		\end{point}

		\begin{response}
			TODO
			\label{re:3.5}
		\end{response}

		\begin{point}
			The sizes of figures are too small.
			\label{pt:3.6}
		\end{point}

		\begin{response}
			We sincerely apologize for the inconvenience. The figures would be amplified in future versions.
			\label{re:3.6}
		\end{response}
	\end{reviewersection}


	\bibliographystyle{IEEEtran}
	\bibliography{IEEEabrv,library.bib}
\end{document}
