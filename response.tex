\documentclass{article}

\usepackage[utf8]{inputenc}
\usepackage{amsfonts}
\usepackage{amsmath}
\usepackage{amssymb}
\usepackage{amsthm}
\usepackage{fullpage}
\usepackage{pdfpages}
\usepackage{siunitx}

\DeclareSIUnit{\belm}{Bm}
\DeclareSIUnit{\dBm}{\deci\belm}
\DeclareSIUnit{\beli}{Bi}
\DeclareSIUnit{\dBi}{\deci\beli}

\newcounter{reviewer}
\setcounter{reviewer}{0}
\newcounter{point}[reviewer]
\setcounter{point}{0}
\newcounter{response}[reviewer]
\setcounter{response}{0}

\let\svbibcite\bibcite
\def\bibcite#1#2{\svbibcite{#1}{R#2}}
\makeatletter
\let\svbiblabel\@biblabel
\def\@biblabel#1{\svbiblabel{R#1}}
\makeatother

\newcommand{\reviewersection}
	{\stepcounter{reviewer} \bigskip \hrule \section*{Reviewer \thereviewer}}

\renewcommand{\thepoint}
	{\thereviewer.\arabic{point}}

\renewcommand{\theresponse}
	{\thereviewer.\arabic{response}}

\newenvironment{point}
	{\refstepcounter{point} \bigskip \noindent {\textbf{Comment~\thepoint} } ---\ \itshape}
	{\par}

\newenvironment{response}
	{\refstepcounter{response} \medskip \noindent \textbf{Response:}\ }
	{\medskip}


\begin{document}
	\includepdf{letter.pdf}

	\begin{reviewersection}
		\begin{point}
			This paper assumed that the perfect channel state information (CSI) of the whole system is available at the base station. However, since the IRS is in general not equipped with a radio frequency chain, the accurate CSI of the reflecting links established by the IRS is very challenging to obtain. The reviewer wonders if the proposed algorithm can also be applied to a case where the imperfect CSI of the network is available at the base station.
			\label{pt:1.1}
		\end{point}

		\begin{response}
			Thank you for pointing out the CSI acquisition issue in presence of IRS. To estimate the cascaded Tx-IRS-Rx link, practical protocols based on element-wise on/off switching [24], joint training sequence and reflection pattern design \cite{Kang2020} and compressed sensing \cite{Wang2020} have been developed for specific networks. The proposed Algorithm 1 relies on such cascaded CSI (not individual Tx-IRS and IRS-Rx channels) at the transmitter (CSIT) and can be readily applied to the quantized CSIT case. However, the codebook design for SWIPT with harvester nonlinearity remains an open issue and one particular challenge is how to evolve from the information codebook to satisfy the power demand.
			\label{re:1.1}
		\end{response}

		\begin{point}
			This paper considered a relatively simple system model where there is only one user in the system. However, in practice, there can be many users co-existing in the SWIPT systems and there may also be two different and independent quality of service requirements: energy harvesting requirement and information decoding requirement. It could be better if the authors can clarify if the proposed algorithm can be employed in a more general case where the base station and the IRS cooperate to serve multiple users in the same time slot.
			\label{pt:1.2}
		\end{point}

		\begin{response}
			We appreciate your suggestion on IRS-aided multi-user SWIPT and believe it would be very promising. To that end, we would first need to tackle the waveform and beamforming design for a multi-user SWIPT system without IRS. Note that in a multi-carrier network, each subcarrier only serves one information user at a time but may serve all energy user simultaneously (use the entire signal for energy purpose), and it involves an additional resource allocation problem that cannot be solved by the GP algorithm proposed in this paper.

			To see the reason, recall that in the single-user scenario, the optimal information beamformer and power beamformer coincide at MRT (see Response \ref{pt:2.2} for details) and the optimal waveform is given by (25), which decouples the design in the spatial domain (beamforming) and frequency domain (power allocation $s_{\mathrm{I/P},n}^2$) to ensure the term involved in the sum rate (7) and DC components (10) -- (13) satisfy $\boldsymbol{h}_n^H\boldsymbol{w}_{\mathrm{I/P},n}=\lVert{\boldsymbol{h}_n}\rVert s_{\mathrm{I/P},n}$. Apparently, picking $s_{\mathrm{I/P}, n}$ as nonnegative real value would not only maximize (27) but also make (28) an amplitude optimization problem and enable GP tools.

			However, this may not be the case in a multi-user scenario. At subband $n$, the composite channel is now $\boldsymbol{h}_{k, n}$ for user $k$ and the term above becomes $\boldsymbol{h}_{k, n}^H\boldsymbol{w}_{\mathrm{I/P},n}$. It is generally complex and require other approaches to solve the highly non-convex optimization problem in the complex field. Although the waveform and beamforming design for multi-user multi-carrier WPT has been investigated in [32], the extension to SWIPT is not straightforward as the power splitting ratio of all users also require a joint update and the SDR method cannot be applied directly.
			\label{re:1.2}
		\end{response}

		\begin{point}
			It is well known that after employing semidefinite programming for handling the phase shift matrix at the IRS, it is very unlikely to obtain a rank-one phase shift matrix without any further modification. The reviewer notices that the authors proposed the Gaussian randomization method to ensure a rank-one solution. Also, the convergence of the proposed overall algorithm also relies on the unit-rank solution. Therefore, to make this paper more comprehensive and convincing, it is suggested to provide more results (such as figures, tables, and data analysis) in the simulation part to show that the rank-one solution can always be obtained even without applying the Gaussian randomization. As this is a very important and interesting conclusion to the colleagues working in the same area, it could be better if the authors can further discuss, interpret, and clarify this in a remark.
			\label{pt:1.3}
		\end{point}

		\begin{response}
			Thank you for raising this issue. The rank-\num{1} property of $\boldsymbol{\Phi}$ to the relaxed problem (24) is suggested by CVX toolbox under different configurations. As indicated by the comment, the strict convergence of the AO algorithm relies on this conclusion (as $\boldsymbol{\phi}$ can be retrieved without performance loss), but it is intricate to provide a mathematical proof due to the coupled objective (24a) and the log sum constraint (24b). Instead, we considered \num{300} individual channel realizations for different $M$, $N$, $L$ and employed \num{30} samples (corresponding to \num{30} problems with different sum rate constraints) for each individual R-E region. The extensive numerical results show that the proposed SCA Algorithm 1 found a unit-rank solution $\boldsymbol{\Phi}$ for all channel realizations, at all rate constraints, and during all iterations. Therefore, the convergence and performance are guaranteed for all tested scenario, and we believe that Algorithm 1 can provide a feasible and reliable passive beamformer in general SWIPT systems.

			[To add a CDF plot of max eigenvalue over sum eigenvalue, which is almost always 1. The simulation is still running.]
			\label{re:1.3}
		\end{response}

		\begin{point}
			The figures in the current version are relatively small, it is suggested to provide larger figures to help the readers better understand the results.
			\label{pt:1.4}
		\end{point}

		\begin{response}
			We sincerely apologize for the inconvenience. The figures would be amplified in future versions.
			\label{re:1.4}
		\end{response}

		\begin{point}
			In the current version, the authors claimed that by applying semidefinite relaxation and omitting the rank-one constraint, the performance loss is negligible. The reviewer wonders if this is because of the relatively simple system model, as there is only one single user who has both power and information requirements. It could be better if the authors can further discuss this issue for a more general multiple user scenario.
			\label{pt:1.5}
		\end{point}

		\begin{response}
			We agree with the reviewer that the performance loss by relaxing the rank-one constraint deserves more attention. In \cite{Feng2021}, we investigated the joint waveform and beamforming design for multi-user multi-carrier WPT (with no information waveform and rate requirements) and formulated the WSE problem as complex constant-modulus QP, enabling an SDR with at least $\gamma = \pi / 4$ approximation to the original objective function. However, due to the reason mentioned in Response \ref{re:1.2}, we did not extend the work to multi-user SWIPT in this paper.

			In contrast, the performance guarantee of rank relaxation in our single-user problem is simply demonstrated by simulation. Due to the coupling of the information and power waveform, the objective function (23) involves
			\begin{equation}
				(t_{\mathrm{I},0}^{(i)} - t_{\mathrm{P},0}^{(i)})^2 = \left(\mathrm{Tr}(\boldsymbol{C}_{\mathrm{I},0}\boldsymbol{\Phi^{(i)}}) - \mathrm{Tr}(\boldsymbol{C}_{\mathrm{P},0}\boldsymbol{\Phi^{(i)}})\right)^2 = \mathrm{Tr}\left((\boldsymbol{C}_{\mathrm{I},0} - \boldsymbol{C}_{\mathrm{P},0}) \boldsymbol{\Phi^{(i)}}\right)^2
			\end{equation}
			which consists of trace square and is not a semidefinite expression. Compared with standard SDP, problem (24) also introduces an additional log sum constraint (24b) for communication purpose, and the existing SDR bounds cannot be applied directly in this case.
			\label{re:1.5}
		\end{response}
	\end{reviewersection}

	\begin{reviewersection}
		\begin{point}
			New RIS models are now being adopted as in [23] where it has been shown that the reflected signals depend on the direction of the arriving signal and this needs to be included in the analysis for realistic quantification.
			\label{pt:2.1}
		\end{point}

		\begin{response}
			Thank you for sharing this acknowledged paper. It investigated the impact of non-zero effective resistance on the reflection pattern and pointed out that the amplitude of the reflection coefficient depends on the phase shift forced on the incoming signal when power dissipation is considered at the IRS. It also proposed an analytical IRS model together with an AO algorithm to maximize the achievable rate by passive beamforming. Simulation results emphasized the importance of modeling such a relationship in practical IRS design. We actually thought about integrating these new models in our system design, but finally decided to use the most common and simplest IRS model in the current stage to reduce the design complexity and provide a primary benchmark for practical IRS-aided SWIPT.
			\label{re:2.1}
		\end{response}

		\begin{point}
			Why is MRT considered as precoder by (25) rather than optimizing it? Is it globally optimal too?
			\label{pt:2.2}
		\end{point}

		\begin{response}
			In the single-user scenario, MRT is the global optimal information and power precoder. We can decouple the waveform in the spatial and frequency domain by
			\begin{equation}
				\boldsymbol{w}_{\mathrm{I/P}, n} = s_{\mathrm{I/P}, n} \bar{\boldsymbol{w}}_{\mathrm{I/P}, n}
			\end{equation}
			where $s_{\mathrm{I/P},n}^2$ denote the power allocated to the information/power waveform at subband $n$, and $\bar{\boldsymbol{w}}_{\mathrm{I/P}, n}$ is the active beamformer at subband $n$. For simplicity, we define $s_{\mathrm{I/P}, n}$ as nonnegative real value (as the phase can be included in the beamformer). From the information perspective, the MRT precoder maximizes $\lvert{\boldsymbol{h}_{n}^H \boldsymbol{w}_{\mathrm{I}, n}}\rvert = s_{\mathrm{I}, n} \lvert{\boldsymbol{h}_{n}^H \bar{\boldsymbol{w}}_{\mathrm{I}, n}}\rvert$ thus maximizes the sum rate (7). From the energy perspective, the MRT precoder maximizes $(\boldsymbol{h}_{n}^H \boldsymbol{w}_{\mathrm{I/P}, n})(\boldsymbol{h}_{n}^H \boldsymbol{w}_{\mathrm{I/P}, n})^*$ thus maximizes the second and fourth order DC terms (10) -- (13). Therefore, MRT is the global optimal information and power precoder. However, this is not the case for the multi-user scenario where the beamformer and power allocation require a joint optimization.
			\label{re:2.2}
		\end{response}

		\begin{point}
			Some strong assumptions like perfect CSI availability limit the practical utility of the proposed analytical results.
			\label{pt:2.3}
		\end{point}

		\begin{response}
			The reviewer is referred to Response \ref{re:1.1}. Indeed, the assumption of perfect CSIT is very ideal and the existing protocols may not provide a good enough estimation in practice. We follow the convention of most existing papers on this topic and expect some breakthroughs in the future research.
			\label{re:2.3}
		\end{response}

		\begin{point}
			All the assumptions and relaxations adopted used in the derivation of results as in (23) need to be explicitly mentioned along with appropriate justification for the same.
			\label{pt:2.4}
		\end{point}

		\begin{response}
			We appreciate your suggestion and have revised the manuscript correspondingly. The original objective function (19) is differentiable and non-concave in $\mathbb{C}^{4N - 2}$, and we approximate (linearize) the second-order terms by the first-order Taylor expansions (20) -- (22) to formulate SCA problems of the original passive beamforming problem, i.e. maximize (19) s.t. (24b) -- (24e). Specifically, the objective affine function (23) is obtained by plugging (20) -- (22) into (19) and it satisfies $\tilde{z}(\boldsymbol{\Phi}^{(i)}, \boldsymbol{\Phi}^{(i)}) = z(\boldsymbol{\Phi}^{(i)}) \ge \tilde{z}(\boldsymbol{\Phi}^{(i)}, \boldsymbol{\Phi}^{(i-1)})$.
			\label{re:2.4}
		\end{response}

		\begin{point}
			Some transformations have been made while solving the original problem, but it has not been explicitly mentioned whether it is equivalent to transformation or not.
			\label{pt:2.5}
		\end{point}

		\begin{response}
			Thank you for the reminder. All transformations are equivalent to their original form and we have make this clear in the revised manuscript.
			\label{re:2.5}
		\end{response}

		\begin{point}
			Are the proposed solutions locally optimal or globally optimal? It is not clear whether the convergence of proposed solution methodologies is local or global? Also, how fast is it?
			\label{pt:2.6}
		\end{point}

		\begin{response}
			Algorithm 1 -- 3 only provide local optimal solutions with local convergence proof, and the performance indeed depends on the initialization. For different sum rate constraints, we use water-filling with MRT to initialize the modulated waveform while use scaled matched filter to initialize the unmodulated waveform (corresponding to a total power of $2P$, regulated afterwards). The phase shift of each IRS element is initialized as i.i.d. uniform random variables over $[0, 2\pi)$. Nevertheless, as the algorithms only converge to local optimal, few R-E points might be strictly worse (in terms of both rate and energy) than the others, especially when $M$ and $L$ are very large. To address this issue, we draw the R-E boundary from the high-rate low-energy (lower right) points to the low-rate high-energy (upper left) points. If a point is strictly dominated, we discard the candidate, reinitialize the waveform and IRS phase shift by the solution at the previous point, then perform the optimization again.

			For a tolerance of $\epsilon=10^{-8}$, the SDR-based Algorithm 1 usually converge within 2 iterations while the GP-based Algorithm 2 tends to converge within 8 iterations. The AO-based Algorithm 3 typically converge within 2 iterations. Unfortunately, the complexity analysis is very challenging as problem (24) and (32) are both in intricate form.
			\label{re:2.6}
		\end{response}

		\begin{point}
			The time complexity of the proposed algorithms, especially involving branch and bound methods, seems to be high especially applications assuming perfectly CSI availability as the coherence times are practically pretty low. So, the authors would like to justify it so that the proposed solution can be obtained over relatively short coherence intervals.
			\label{pt:2.7}
		\end{point}

		\begin{response}
			We agree that the proposed algorithm (especially Algorithm 2) could be very time-consuming for applications with short coherence time. The proposed algorithms ensure optimal performance by iteratively solving intricate optimization problems. Although the effectiveness of suboptimal design in closed form (e.g. scaled matched filter) has been validated in practical WPT systems, the extension to SWIPT is much more challenging because both waveforms and the splitting ratio need to be balanced under the influence of SNR.
			\label{re:2.7}
		\end{response}

		\begin{point}
			How practical is it to consider lossless reflection from the RIS? Specifically, by considering the magnitude to be 1, the reflection losses at the RIS have been ignored.
			\label{pt:2.8}
		\end{point}

		\begin{response}
			Green's decomposition suggests that the backscattered/reflected signal of an antenna can be decomposed into the \emph{structural mode} component and the \emph{antenna mode} component. The former is fixed for a given antenna and can be regarded as part of the environment multipath, while the latter is adjustable and depends on the mismatch of the antenna and load impedances. The reflection coefficient of the $l$-th IRS element is thus defined as
			\begin{equation}
				\phi_l = \frac{Z_l - Z_0}{Z_l + Z_0}
			\end{equation}
			where $Z_0$ is the characteristic impedance and $Z_l = R_l + j X_l$ is the impedance of the $l$-th IRS element. It implies that $\lvert{\phi_l}\rvert \le 1$ for $R_l \ge 0$, and we assume $Z_l$ is pure reactive such that
			\begin{equation}
				\phi_l = \frac{j X_l - Z_0}{j X_l + Z_0} = e^{j \theta_{l}}
			\end{equation}
			which corresponds to the lossless reflection model adopted in the paper. Nevertheless, due to the non-zero power consumption at the IRS in practice, $R_l$ would be positive and the reflection coefficient $\phi_l$ not only has a magnitude smaller than 1 but also depends on the phase shift. [23] found that $\lvert{\phi}_l\rvert$ experiences a trough as low as \num{0.2} for $R_l = 2.5\,\si{\ohm}$ at zero phase shift, which suggests that the position and direction of the IRS should be carefully designed.
			\label{re:2.8}
		\end{response}

		\begin{point}
			Minor comment: The size of all the numerical results figures is too small.
			\label{pt:2.9}
		\end{point}

		\begin{response}
			We sincerely apologize for the inconvenience. The figures would be amplified in future versions.
			\label{re:2.9}
		\end{response}
	\end{reviewersection}

	\begin{reviewersection}
		\begin{point}
			First of all, motivations of studying the IRS on SWIPT is very unclear to me. Please clarify.
			\label{pt:3.1}
		\end{point}

		\begin{response}
			We appreciate your suggestion and have modified the manuscript to emphasize this point. A major challenge for SWIPT is that the information decoder and energy harvester have different sensitivity -- although conventional radios as Wi-Fi and Bluetooth can support a signal strength of \num{-90} to \SI{-100}{\dBm}, most existing harvesters only capture signals at \num{-20} to \SI{-30}{\dBm}. Since the transmit power is strictly subject to regulations, it is crucial to increase the energy efficiency to boost the signal strength and extend the system coverage. Therefore, we believe the effective channel enhancement and the low power consumption of IRS is a perfect match for SWIPT.
			\label{re:3.1}
		\end{response}

		\begin{point}
			Also, the contributions of this work are rather unclear, and thus, those should be better mentioned.
			\label{pt:3.2}
		\end{point}

		\begin{response}
			Thank you for the opinion and the manuscript has been revised accordingly. This paper proposed an IRS-aided SWIPT network and considered a joint waveform and beamforming design over spatial and frequency domains under energy harvester nonlinearity. To the best of our knowledge, relevant existing papers assumed linear harvester model where the RF-to-DC conversion efficiency is a constant regardless of its input. However, the harvester consists of an antenna to intercept RF signal and a diode rectifier (nonlinear device) to produce DC power, and the RF-to-DC efficiency indeed depends on the power and shape of the input waveform. By exploiting the joint impact of passive beamforming and harvester nonlinearity, we found that:
			\begin{itemize}
				\item the IRS should be developed next to the transmitter or the receiver due to double fading;
				\item the optimal transceiving mode depends on the SNR and the number of subbands;
				\item the power scaling laws of active and passive beamforming are in the order of $M^2$ and $L^4$, respectively;
				\item the IRS reflection coefficient cannot be designed per subband and there exists a tradeoff in subchannel enhancement;
				\item the optimal passive beamforming can be approximated in closed form for narrowband transmission but varies at different R-E points for broadband transmission.
			\end{itemize}
			\label{re:3.2}
		\end{response}

		\begin{point}
			Please explain the derived results more intuitively for better understanding.
			\label{pt:3.3}
		\end{point}

		\begin{response}
			TODO
			\label{re:3.3}
		\end{response}

		\begin{point}
			Authors assumed the unrealistic situation: the channels are assumed to be perfectly known. However, in practice, the channel should be estimated, e.g., as studied in \cite{Kang2020}, \cite{You2019}. It would be much better to discuss the channel estimation issue by citing the above references.
			\label{pt:3.4}
		\end{point}

		\begin{response}
			The reviewer is referred to Response \ref{re:1.1}. We have incorporated the channel estimation issue into the revised manuscript by citing those references.
			\label{re:3.4}
		\end{response}

		\begin{point}
			More simulation results should be added to better and aggregately validate the effectiveness of the proposed method.
			\label{pt:3.5}
		\end{point}

		\begin{response}
			TODO
			\label{re:3.5}
		\end{response}

		\begin{point}
			The sizes of figures are too small.
			\label{pt:3.6}
		\end{point}

		\begin{response}
			We sincerely apologize for the inconvenience. The figures would be amplified in future versions.
			\label{re:3.6}
		\end{response}
	\end{reviewersection}


	\bibliographystyle{IEEEtran}
	\bibliography{IEEEabrv,library.bib}
\end{document}
